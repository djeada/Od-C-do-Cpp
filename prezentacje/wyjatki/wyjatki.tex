%!TeX encoding = UTF-8
%!TeX program = xelatex
\documentclass[notheorems, aspectratio=54]{beamer}
% aspectratio: 1610, 149, 54, 43(default), 32
\usepackage[utf8]{inputenc} % `utf8` option to match Editor encoding
\usepackage[T1]{fontenc}
\usepackage{latexsym}
\usepackage{amsmath,amssymb}
\usepackage{mathtools}
\usepackage{color,xcolor}
\usepackage{graphicx}
\usepackage{algorithm}
\usepackage{amsthm}
\usepackage{lmodern} % 解决 font warning
% \usepackage[UTF8]{ctex}
\usepackage{animate} % insert gif
\usepackage{listings}

\usepackage{karnaugh-map}

\usetikzlibrary{matrix,calc}
\usepackage{lipsum} % To generate test text 
\usepackage{ulem} % 下划线,波浪线

\usepackage{listings} % display code on slides; don't forget [fragile] option after \begin{frame}


% ----------------------------------------------
% tikx
\usepackage{framed}
\usepackage{tikz}
\usepackage{pgf}
\usetikzlibrary{calc,trees,positioning,arrows,chains,shapes.geometric,%
    decorations.pathreplacing,decorations.pathmorphing,shapes,%
    matrix,shapes.symbols}
\pgfmathsetseed{1} % To have predictable results
% Define a background layer, in which the parchment shape is drawn
\pgfdeclarelayer{background}
\pgfsetlayers{background,main}

% define styles for the normal border and the torn border
\tikzset{
  normal border/.style={black!70!gray, decorate, 
     decoration={random steps, segment length=2.5cm, amplitude=.7mm}},
  torn border/.style={black!70!gray, decorate, 
     decoration={random steps, segment length=.5cm, amplitude=1.7mm}}}

% Macro to draw the shape behind the text, when it fits completly in the
% page
\def\parchmentframe#1{
\tikz{
  \node[inner sep=2em] (A) {#1};  % Draw the text of the node
  \begin{pgfonlayer}{background}  % Draw the shape behind
  \fill[normal border] 
        (A.south east) -- (A.south west) -- 
        (A.north west) -- (A.north east) -- cycle;
  \end{pgfonlayer}}}

% Macro to draw the shape, when the text will continue in next page
\def\parchmentframetop#1{
\tikz{
  \node[inner sep=2em] (A) {#1};    % Draw the text of the node
  \begin{pgfonlayer}{background}    
  \fill[normal border]              % Draw the ``complete shape'' behind
        (A.south east) -- (A.south west) -- 
        (A.north west) -- (A.north east) -- cycle;
  \fill[torn border]                % Add the torn lower border
        ($(A.south east)-(0,.2)$) -- ($(A.south west)-(0,.2)$) -- 
        ($(A.south west)+(0,.2)$) -- ($(A.south east)+(0,.2)$) -- cycle;
  \end{pgfonlayer}}}

% Macro to draw the shape, when the text continues from previous page
\def\parchmentframebottom#1{
\tikz{
  \node[inner sep=2em] (A) {#1};   % Draw the text of the node
  \begin{pgfonlayer}{background}   
  \fill[normal border]             % Draw the ``complete shape'' behind
        (A.south east) -- (A.south west) -- 
        (A.north west) -- (A.north east) -- cycle;
  \fill[torn border]               % Add the torn upper border
        ($(A.north east)-(0,.2)$) -- ($(A.north west)-(0,.2)$) -- 
        ($(A.north west)+(0,.2)$) -- ($(A.north east)+(0,.2)$) -- cycle;
  \end{pgfonlayer}}}

% Macro to draw the shape, when both the text continues from previous page
% and it will continue in next page
\def\parchmentframemiddle#1{
\tikz{
  \node[inner sep=2em] (A) {#1};   % Draw the text of the node
  \begin{pgfonlayer}{background}   
  \fill[normal border]             % Draw the ``complete shape'' behind
        (A.south east) -- (A.south west) -- 
        (A.north west) -- (A.north east) -- cycle;
  \fill[torn border]               % Add the torn lower border
        ($(A.south east)-(0,.2)$) -- ($(A.south west)-(0,.2)$) -- 
        ($(A.south west)+(0,.2)$) -- ($(A.south east)+(0,.2)$) -- cycle;
  \fill[torn border]               % Add the torn upper border
        ($(A.north east)-(0,.2)$) -- ($(A.north west)-(0,.2)$) -- 
        ($(A.north west)+(0,.2)$) -- ($(A.north east)+(0,.2)$) -- cycle;
  \end{pgfonlayer}}}

% Define the environment which puts the frame
% In this case, the environment also accepts an argument with an optional
% title (which defaults to ``Example'', which is typeset in a box overlaid
% on the top border
\newenvironment{parchment}[1][Example]{%
  \def\FrameCommand{\parchmentframe}%
  \def\FirstFrameCommand{\parchmentframetop}%
  \def\LastFrameCommand{\parchmentframebottom}%
  \def\MidFrameCommand{\parchmentframemiddle}%
  \vskip\baselineskip
  \MakeFramed {\FrameRestore}
  \noindent\tikz\node[inner sep=1ex, draw=black!20, fill=black!90, 
          anchor=west, overlay] at (0em, 2em) {\sffamily#1};\par}%
{\endMakeFramed}

% ----------------------------------------------

\mode<presentation>{
    \usetheme{Warsaw}
    % Boadilla CambridgeUS
    % default Antibes Berlin Copenhagen
    % Madrid Montpelier Ilmenau Malmoe
    % Berkeley Singapore Warsaw
    \usecolortheme{seagull}
    % beetle, beaver, orchid, whale, dolphin, seagull
    \useoutertheme{infolines}
    % infolines miniframes shadow sidebar smoothbars smoothtree split tree
    \useinnertheme{circles}
    % circles, rectanges, rounded, inmargin
}

% ---------------------------------------------------------------------
% Jet Black Theme
\setbeamercolor{normal text}{fg=white,bg=black!90}
\setbeamercolor{structure}{fg=white}

\setbeamercolor{alerted text}{fg=red!85!black}

\setbeamercolor{item projected}{use=item,fg=black,bg=item.fg!35}

\setbeamercolor*{palette primary}{use=structure,fg=structure.fg}
\setbeamercolor*{palette secondary}{use=structure,fg=structure.fg!95!black}
\setbeamercolor*{palette tertiary}{use=structure,fg=structure.fg!90!black}
\setbeamercolor*{palette quaternary}{use=structure,fg=structure.fg!95!black,bg=black!80}

\setbeamercolor*{framesubtitle}{fg=white}

\setbeamercolor*{block title}{parent=structure,bg=black!70!gray}
\setbeamercolor*{block body}{fg=black,bg=black!10}
\setbeamercolor*{block title alerted}{parent=alerted text,bg=black!15}
\setbeamercolor*{block title example}{parent=example text,bg=black!15}
% ---------------------------------------------------------------------


% ---------------------------------------------------------------------
% flow chart
\tikzset{
    >=stealth',
    punktchain/.style={
        rectangle, 
        rounded corners, 
        % fill=black!10,
        draw=white, very thick,
        text width=6em,
        minimum height=2em, 
        text centered, 
        on chain
    },
    largepunktchain/.style={
        rectangle,
        rounded corners,
        draw=white, very thick,
        text width=10em,
        minimum height=2em,
        on chain
    },
    line/.style={draw, thick, <-},
    element/.style={
        tape,
        top color=white,
        bottom color=blue!50!black!60!,
        minimum width=6em,
        draw=blue!40!black!90, very thick,
        text width=6em, 
        minimum height=2em, 
        text centered, 
        on chain
    },
    every join/.style={->, thick,shorten >=1pt},
    decoration={brace},
    tuborg/.style={decorate},
    tubnode/.style={midway, right=2pt},
    font={\fontsize{10pt}{12}\selectfont},
}
% ---------------------------------------------------------------------

% code setting
\lstset{
    language=C++,
    basicstyle=\ttfamily\footnotesize,
    keywordstyle=\color{red},
    breaklines=true,
    xleftmargin=2em,
    numbers=left,
    numberstyle=\color[RGB]{222,155,81},
    frame=leftline,
    tabsize=4,
    breakatwhitespace=false,
    showspaces=false,               
    showstringspaces=false,
    showtabs=false,
    morekeywords={Str, Num, List},
}

% ---------------------------------------------------------------------

\newcommand{\reditem}[1]{\setbeamercolor{item}{fg=red}\item #1}

% 缩放公式大小
\newcommand*{\Scale}[2][4]{\scalebox{#1}{\ensuremath{#2}}}

% 解决 font warning
\renewcommand\textbullet{\ensuremath{\bullet}}

% -------------------------------------------------------------

%% preamble
\title[Wyjątki]{Wyjątki}
% \subtitle{The subtitle}
\author{Adam Djellouli}

% -------------------------------------------------------------

\begin{document}

% title frame
\begin{frame}
    \titlepage
\end{frame}

\begin{frame}

Pytania, na które checemy odpowiedzieć:

\begin{itemize}
\item Czym to jest?
\item Jak to działa?
\item Po co to komu potrzebne?
\item Kiedy się tego używa?
\end{itemize}

\end{frame}

\begin{frame}
O co w tym chodzi?\\

Każdy program działa jedynie gdy spełnione są określone założenia.\\
Część z tych założeń jest całkowicie niezależnych od programisty.\\~\\
Co ma się stać gdy, któreś z założeń nie zostanie spełnione?\\

\begin{enumerate}
\item Zakończenie działania programu.
\item Zwrócenie kodu błędu.
\item Wyrzucenie wyjątku.
\end{enumerate}

\end{frame}

\begin{frame}

Mówiąc o wyjątkach, mamy najczęściej na myśli jedną z dwóch rzeczy:

\begin{enumerate}
\item Zdarzenie mające miejsce w czasie działania programu i uniemożliwiające działanie programu.
\item Mechanizm będący częścią języka programowania, który pozwala na radzenie sobie z tymi zdarzeniami.
\end{enumerate}

\end{frame}

\begin{frame}
Jak działa ten mechanizm?

\begin{lstlisting}[language=C++]
#include <iostream>

void funkcja()
{
    std::cout << "Przed wyrzuceniem wyjatku" << std::endl;
    throw std::runtime_error("Wyjatek");
    std::cout << "Po wyrzuceniu wyjatku" << std::endl;
}

int main()
{
    funkcja();
    std::cout << "Po wywolaniu funkcji" << std::endl;
    return 0;
}
\end{lstlisting}
\end{frame}

\begin{frame}
Obsługa wyjątków

\begin{lstlisting}[language=C++]
#include <iostream>

void funkcja()
{
    throw std::runtime_error("Wyjatek");
}

int main() {
  try
  {
      funkcja();
  }
  catch(std::runtime_error& e)
  {
      std::cout << "Wyjątek: " << e.what() << std::endl;
  }

  std::cout << "Zycie toczy sie dalej" << std::endl;

  return 0;
}
\end{lstlisting}
\end{frame}

\begin{frame}
Biblioteka standardowa używa wyjątków.

Przykład std::out_of_range:
\begin{lstlisting}[language=C++]
int main() {
  std::vector wektor{3, 5, 2};

  std::cout << wektor.at(0) << std::endl;
  std::cout << wektor.at(-1) << std::endl;

  return 0;
}
\end{lstlisting}

Przykład std::invalid_argument:
\begin{lstlisting}[language=C++]
int main() {

  // bitset spodziewa sie napisu zlozonego wylacznie
  // z 0 lub 1.
  std::bitset<8> przyklad {std::string("01234")};

  return 0;
}
\end{lstlisting}

\end{frame}

\begin{frame}

Kiedy użyć wyjątków?

W sytuacji gdy wykonywanie zadania przez funkcję uniemożliwone jest przez zewnętrzną przyczynę. Przykłady:

\begin{itemize}
\item Wartości parametrów nie spełniają założeń. Nie ma możliwości na poprawne rozwiązanie problemu.
\item Brak miejsca na dysku. Program chciałby użyć więcej pamięci niż jest dostępne.
\item Serwer nie odpowiada (funkcja mająca zwrócić dane z serwera).
\item Format danych w bazie danych jest niezgodny z wymaganiami programu (funkcja mająca zwrócić dane w określonym formacie).
\end{itemize}

\end{frame}
\begin{frame}

Jak nie używać wyjątków?

\begin{itemize}
\item Nie należy wyjątków jak skoków.

\begin{lstlisting}[language=C++]

  bool czyLiczba(std::string napis) {
    try {
      std::stoi(napis);
      return true;
    } catch (std::invalid_argument& e) {
      return false;
    }
  }
\end{lstlisting}
  
\item Nie powinno się wyrzucać wyjątków gdy sytuacja tego nie wymaga.

Przykład: wyrzucenie wyjątku gdy użytkownik podaje niepoprawne hasło.
Albo w aplikacji kalkulator użytkownik chce dzielić przez 0.
Wszystko zależy od poziomu w aplikacji na którym operujemy.

\item Użycie złego typu wyjątku.

\begin{lstlisting}[language=C++]
void funkcja() {
  throw std::invalid_argument("Wyjatek");
}
\end{lstlisting}

\item Używanie mylącego komunikatu.

\begin{lstlisting}[language=C++]
void funkcja() {
  throw std::exception("Wystapil problem");
}
\end{lstlisting}

\item Połykanie wyjątków.

\begin{lstlisting}[language=C++]
try {
  funkcja();
} catch (std::exception& e) {
  return;
}
\end{lstlisting}

\end{itemize}

\end{frame}

\begin{frame}
 Dlaczego wyjątki?

Wiemy już, że funkcje mogą napotkać na sytuacje uniemożliwiające ich poprawne działanie.
Jakie mamy alternatywy do radzenia sobie z tym problemem?

  \begin{itemize}
  \item można użyć zmiennych jako flag i ustawiać je w chwilii pojawienia się problemu;
  \item w funkcji zwracać wartość, która sugerowałaby, że pojawił się błąd (np. -1);
  \item użyć wyjątków;
  \end{itemize}
  
  Obie z dwóch pierwszych opcji są złe chociażby dlatego, że wymagają od nas ciągłego sprawdzania czy nie pojawiła się problematyczna sytuacja.
  Łatwo jest zignorować błędy w takiej sytuacji i ciężko jest oddzielić kod obsługi błędu od reszty kodu (używamy te same mechanizmy).
  Zamiast tego wyjątki umożliwiają wyraźne oddzielenie kodu wywoływanego w normalnej sytuacji, od kodu wywoływanego w sytuacji wyjątkowej.
  Dodatkowo wyjątki zapewniają uwolnienie wszystkich zasobów dzięki destruktorom.
  \end{frame}


\begin{frame}
Wyjatki schowane (o uwalnianiu pamięci)

1. Funkcja napotyka problematyczną sytuację, więc wyrzuca wyjątek, który niesie informacje o tej sytuacji.\\
2. Wyjątek wyrzucony jest do miejsca wywołania funkcji.\\
3. Wyjątek może tam zostać złapany i obsłużony lub rzucony dalej.\\
4. Podczas tego procesu rzucania wyjątkiem ma miejsce coś niezwykle isotnego. Pamięć zarezerwowana na stercie dla obiektów tworzonych w tych funkcjach jest uwalniana!
\end{frame}

\begin{frame}[fragile]
Wyjątki w konstruktorze\\
\begin{lstlisting}

Piesek::Piesek 
{
	wsk_na_kosc = new popcorn();
	movie_ptr = new movie(bigMovie);
	soda_ptr = new soda();
}
\end{lstlisting}

Co jeśli pojawi się problem w trakcie tworzenia obiektu? 
Wiemy, że wyjątki zapewniają uwolnienie zarezerwowanych zasobów dzięki destrukotrom.
Ale jak wywołać destruktor dla częściowo zainicjalizowanego obiektu?
Która część konstrukotra jest już gotowa?

\end{frame}

\begin{frame}
Wyjątek w trakcie obsługi wyjątku?\\

Jakie mamy opcje? Możemy:
\begin{itemize}
\item Zignorować drugi wyjąt i kontynuować obsługę pierwszego.
\item Porzucić obsługę pierwszego wyjątku i przejść do obsługi drugiego.
\item Poddać się i zamknąć program.
\end{itemize}

Twórcy C++ wybrali trzecią opcję. 
Jeśli wyrzucisz wyjątek w trakcie obsługi innego wyjątku, to wywołana zostanie funkcja std::terminate().

\end{frame}

\begin{frame}
  Bibliografia:\\
  \begin{itemize}
  \item http://www.cplusplus.com/doc/tutorial/exceptions/
  \item https://en.cppreference.com/w/cpp/language/exceptions
  \item https://github.com/isocpp/CppCoreGuidelines/blob/master/CppCoreGuidelines.md#exceptions
  \end{itemize}
\end{document}