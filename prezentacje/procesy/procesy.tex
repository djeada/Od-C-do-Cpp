%!TeX encoding = UTF-8
%!TeX program = xelatex
\documentclass[notheorems, aspectratio=54]{beamer}
% aspectratio: 1610, 149, 54, 43(default), 32
\usepackage[utf8]{inputenc} % `utf8` option to match Editor encoding
\usepackage[T1]{fontenc}
\usepackage{latexsym}
\usepackage{amsmath,amssymb}
\usepackage{mathtools}
\usepackage{color,xcolor}
\usepackage{graphicx}
\usepackage{algorithm}
\usepackage{amsthm}
\usepackage{lmodern} % 解决 font warning
% \usepackage[UTF8]{ctex}
\usepackage{animate} % insert gif
\usepackage{listings}

\usepackage{karnaugh-map}

\usetikzlibrary{matrix,calc}
\usepackage{lipsum} % To generate test text 
\usepackage{ulem} % 下划线,波浪线

\usepackage{listings} % display code on slides; don't forget [fragile] option after \begin{frame}


% ----------------------------------------------
% tikx
\usepackage{framed}
\usepackage{tikz}
\usepackage{pgf}
\usetikzlibrary{calc,trees,positioning,arrows,chains,shapes.geometric,%
    decorations.pathreplacing,decorations.pathmorphing,shapes,%
    matrix,shapes.symbols}
\pgfmathsetseed{1} % To have predictable results
% Define a background layer, in which the parchment shape is drawn
\pgfdeclarelayer{background}
\pgfsetlayers{background,main}

% define styles for the normal border and the torn border
\tikzset{
  normal border/.style={black!70!gray, decorate, 
     decoration={random steps, segment length=2.5cm, amplitude=.7mm}},
  torn border/.style={black!70!gray, decorate, 
     decoration={random steps, segment length=.5cm, amplitude=1.7mm}}}

% Macro to draw the shape behind the text, when it fits completly in the
% page
\def\parchmentframe#1{
\tikz{
  \node[inner sep=2em] (A) {#1};  % Draw the text of the node
  \begin{pgfonlayer}{background}  % Draw the shape behind
  \fill[normal border
        (A.south east) -- (A.south west) -- 
        (A.north west) -- (A.north east) -- cycle;
  \end{pgfonlayer}}}

% Macro to draw the shape, when the text will continue in next page
\def\parchmentframetop#1{
\tikz{
  \node[inner sep=2em] (A) {#1};    % Draw the text of the node
  \begin{pgfonlayer}{background}    
  \fill[normal border]              % Draw the ``complete shape'' behind
        (A.south east) -- (A.south west) -- 
        (A.north west) -- (A.north east) -- cycle;
  \fill[torn border]                % Add the torn lower border
        ($(A.south east)-(0,.2)$) -- ($(A.south west)-(0,.2)$) -- 
        ($(A.south west)+(0,.2)$) -- ($(A.south east)+(0,.2)$) -- cycle;
  \end{pgfonlayer}}}

% Macro to draw the shape, when the text continues from previous page
\def\parchmentframebottom#1{
\tikz{
  \node[inner sep=2em] (A) {#1};   % Draw the text of the node
  \begin{pgfonlayer}{background}   
  \fill[normal border]             % Draw the ``complete shape'' behind
        (A.south east) -- (A.south west) -- 
        (A.north west) -- (A.north east) -- cycle;
  \fill[torn border]               % Add the torn upper border
        ($(A.north east)-(0,.2)$) -- ($(A.north west)-(0,.2)$) -- 
        ($(A.north west)+(0,.2)$) -- ($(A.north east)+(0,.2)$) -- cycle;
  \end{pgfonlayer}}}

% Macro to draw the shape, when both the text continues from previous page
% and it will continue in next page
\def\parchmentframemiddle#1{
\tikz{
  \node[inner sep=2em] (A) {#1};   % Draw the text of the node
  \begin{pgfonlayer}{background}   
  \fill[normal border]             % Draw the ``complete shape'' behind
        (A.south east) -- (A.south west) -- 
        (A.north west) -- (A.north east) -- cycle;
  \fill[torn border]               % Add the torn lower border
        ($(A.south east)-(0,.2)$) -- ($(A.south west)-(0,.2)$) -- 
        ($(A.south west)+(0,.2)$) -- ($(A.south east)+(0,.2)$) -- cycle;
  \fill[torn border]               % Add the torn upper border
        ($(A.north east)-(0,.2)$) -- ($(A.north west)-(0,.2)$) -- 
        ($(A.north west)+(0,.2)$) -- ($(A.north east)+(0,.2)$) -- cycle;
  \end{pgfonlayer}}}

% Define the environment which puts the frame
% In this case, the environment also accepts an argument with an optional
% title (which defaults to ``Example'', which is typeset in a box overlaid
% on the top border
\newenvironment{parchment}[1][Example]{%
  \def\FrameCommand{\parchmentframe}%
  \def\FirstFrameCommand{\parchmentframetop}%
  \def\LastFrameCommand{\parchmentframebottom}%
  \def\MidFrameCommand{\parchmentframemiddle}%
  \vskip\baselineskip
  \MakeFramed {\FrameRestore}
  \noindent\tikz\node[inner sep=1ex, draw=black!20, fill=black!90, 
          anchor=west, overlay] at (0em, 2em) {\sffamily#1};\par}%
{\endMakeFramed}

% ----------------------------------------------

\mode<presentation>{
    \usetheme{Warsaw}
    % Boadilla CambridgeUS
    % default Antibes Berlin Copenhagen
    % Madrid Montpelier Ilmenau Malmoe
    % Berkeley Singapore Warsaw
    \usecolortheme{seagull}
    % beetle, beaver, orchid, whale, dolphin, seagull
    \useoutertheme{infolines}
    % infolines miniframes shadow sidebar smoothbars smoothtree split tree
    \useinnertheme{circles}
    % circles, rectanges, rounded, inmargin
}

% ---------------------------------------------------------------------
% Jet Black Theme
\setbeamercolor{normal text}{fg=white,bg=black!90}
\setbeamercolor{structure}{fg=white}

\setbeamercolor{alerted text}{fg=red!85!black}

\setbeamercolor{item projected}{use=item,fg=black,bg=item.fg!35}

\setbeamercolor*{palette primary}{use=structure,fg=structure.fg}
\setbeamercolor*{palette secondary}{use=structure,fg=structure.fg!95!black}
\setbeamercolor*{palette tertiary}{use=structure,fg=structure.fg!90!black}
\setbeamercolor*{palette quaternary}{use=structure,fg=structure.fg!95!black,bg=black!80}

\setbeamercolor*{framesubtitle}{fg=white}

\setbeamercolor*{block title}{parent=structure,bg=black!70!gray}
\setbeamercolor*{block body}{fg=black,bg=black!10}
\setbeamercolor*{block title alerted}{parent=alerted text,bg=black!15}
\setbeamercolor*{block title example}{parent=example text,bg=black!15}
% ---------------------------------------------------------------------


% ---------------------------------------------------------------------
% flow chart
\tikzset{
    >=stealth',
    punktchain/.style={
        rectangle, 
        rounded corners, 
        % fill=black!10,
        draw=white, very thick,
        text width=6em,
        minimum height=2em, 
        text centered, 
        on chain
    },
    largepunktchain/.style={
        rectangle,
        rounded corners,
        draw=white, very thick,
        text width=10em,
        minimum height=2em,
        on chain
    },
    line/.style={draw, thick, <-},
    element/.style={
        tape,
        top color=white,
        bottom color=blue!50!black!60!,
        minimum width=6em,
        draw=blue!40!black!90, very thick,
        text width=6em, 
        minimum height=2em, 
        text centered, 
        on chain
    },
    every join/.style={->, thick,shorten >=1pt},
    decoration={brace},
    tuborg/.style={decorate},
    tubnode/.style={midway, right=2pt},
    font={\fontsize{10pt}{12}\selectfont},
}
% ---------------------------------------------------------------------

% code setting
\lstset{
    language=C++,
    basicstyle=\ttfamily\footnotesize,
    keywordstyle=\color{red},
    breaklines=true,
    xleftmargin=2em,
    numbers=left,
    numberstyle=\color[RGB]{222,155,81},
    frame=leftline,
    tabsize=4,
    breakatwhitespace=false,
    showspaces=false,               
    showstringspaces=false,
    showtabs=false,
    morekeywords={Str, Num, List},
}

% ---------------------------------------------------------------------

\newcommand{\reditem}[1]{\setbeamercolor{item}{fg=red}\item #1}

% 缩放公式大小
\newcommand*{\Scale}[2][4]{\scalebox{#1}{\ensuremath{#2}}}

% 解决 font warning
\renewcommand\textbullet{\ensuremath{\bullet}}

% -------------------------------------------------------------

%% preamble
\title[Procesy]{Procesy}
% \subtitle{The subtitle}
\author{Adam Djellouli}

% -------------------------------------------------------------

\begin{document}

% title frame
\begin{frame}
    \titlepage
\end{frame}

\begin{frame}

W Linuxie mamy dostępne dwie funkcje do tworzenia nowych procesów w programie:

\begin{enumerate}
\item fork() - tworzy nowy proces będący kopią procesu wywołującego. 
\item exec() - tworzy nowy proces, który podmienia proces wywołujący.
\end{enumerate}

\end{frame}

\begin{frame}
Fork
Fork klonuje aktualnie wykonywany proces i zwraca liczbę. 
Liczba ta istnieje zarówno dla procesu wywołującego, jak i dla procesu potomnego!

\begin{itemize}
\item rodzic otrzymuje PID procesu potomnego.
\item proces potomny otrzymuje 0.
\end{itemize}

Dziecko i rodzic są identycznymi procesami. Co to znaczy?

\begin{itemize} 
\item Dziecko ma ten sam ciąg wywołań funkcji.
\item Dziecko i rodzic współdzielą zasoby takie jak otwarte pliki, urządzenia, gniazda itd.
\item Procesy mają różną pamięć wirtualną.
\item Procesy mają różne PID.
\end{frame}

\begin{frame}
Exec podmienia aktuanie wykonywany program na inny. Co to znaczy?

\begin{itemize}
\item Oryginalny program przestaje być wykonywany.
\item Stan oryginalnego procesu jest utracony.
\item Nowy program otrzymuje PID procesu wywołującego.
\item Nowy program dziedziczy zasoby procesu wywołującego.
\end{itemize}

Przykładem funkcji z rodziny exec jest execv().

\begin{lstlisting}[language=C++]
int execv(const char *path, char *const argv[]);
\end{lstlisting}

Pierwszym argumentem jest ścieżka do pliku, który ma zostać uruchomiony. 
Drugim argumentem jest tablica wskaźników na argumenty nowego programu.
Ostatnim elementem tablicy jest zawsze wskaźnik na NULL.
\end{frame}

\begin{frame}

Procesy

\begin{itemize}
\item Procesy to wykonywane programy.
\item Wykonywany program ładowany jest do pamięci wirtualnej.
\item Każdy proces ma przydzieloną przestrzeń adresów w pamięci wirtualnej.
\item Procesy są zorganizowane hierarchicznie, każdy proces ma jednego rodzica.
\item Procesy mają swoje własne zasoby, takie jak otwarte pliki, gniazda itd.
\item Procesy identyfikowane są przez unikalną liczbę, zwaną PID.
\item Wiele procesów może wykonywać ten sam program.
\item Każdy ma oddzielną przestrzeń w pamięci wirtualnej.
\end{frame}


\begin{frame}
Wywołanie systemowe

\begin{itemize}
\item Wywołania systemowe pozwalają procesom na komunikację z jądrem systemu. 
\item Zazwyczaj nie umieszczamy wywołań systemowych bezpośrednio w kodzie programu.
\item Istnieją biblioteki oferujące przyjazne nakładki do wywołań systemowych.
\end{itemize}

Przykładem jest funkcja getpid(), zwracająca PID procesu. Innym przykładem jest funkcja 
getppid(), zwracająca PID rodzica procesu.

\end{frame}

\begin{frame}

Przykład funkcji wykorzystującej wywołania systemowe:

\begin{lstlisting}[language=C++]
#include <stdio.h> 
#include <stdlib.h> 
#include <sys/types.h> 
#include <unistd.h> 

int main() { 
  printf("PID procesu: %d \n", getpid()); 
  printf("PID rodzica: %d \n", getppid());
  printf("Cross verification of pid's by executing process commands on shell \n"); 
  system("ps -ef"); 
  return 0; 
}
\end{lstlisting}

\end{frame}

\begin{frame}
Komunikacja międzyprocesowa (IPC)

\begin{itemize}
\item Ogólnie możemy podzielić procesy na dwie kategorie.
\item Procesy działające w odosobnieniu od innych procesów oraz procesy współdziałające z innymi procesami.
\item Istnieją różne techniki na umożliwienie wymiany danych między procesami.
\item Przykładowo procesy mogą przekazywać sobie dane za pomocą plików (najstarsza i najprostsza forma).
\item Procesy mogą również dzielić pamięć wirtualną.
\item Sygnały mogą być wysyłane między procesami.
\item Potoki pozwalają na przekazanie wyniku działania jednego programu, jako dane wejściowe do innego programu.
\item Kolejka komunikatów pozwala na przesyłanie wiadomości między procesami.
\end{itemize}

\end{frame}

\end{document}
