%!TeX encoding = UTF-8
%!TeX program = xelatex
\documentclass[notheorems, aspectratio=54]{beamer}
% aspectratio: 1610, 149, 54, 43(default), 32
\usepackage[utf8]{inputenc} % `utf8` option to match Editor encoding
\usepackage[T1]{fontenc}
\usepackage{latexsym}
\usepackage{amsmath,amssymb}
\usepackage{mathtools}
\usepackage{color,xcolor}
\usepackage{graphicx}
\usepackage{algorithm}
\usepackage{amsthm}
\usepackage{lmodern} % 解决 font warning
% \usepackage[UTF8]{ctex}
\usepackage{animate} % insert gif
\usepackage{listings}

\usepackage{karnaugh-map}

\usetikzlibrary{matrix,calc}
\usepackage{lipsum} % To generate test text 
\usepackage{ulem} % 下划线,波浪线

\usepackage{listings} % display code on slides; don't forget [fragile] option after \begin{frame}


% ----------------------------------------------
% tikx
\usepackage{framed}
\usepackage{tikz}
\usepackage{pgf}
\usetikzlibrary{calc,trees,positioning,arrows,chains,shapes.geometric,%
    decorations.pathreplacing,decorations.pathmorphing,shapes,%
    matrix,shapes.symbols}
\pgfmathsetseed{1} % To have predictable results
% Define a background layer, in which the parchment shape is drawn
\pgfdeclarelayer{background}
\pgfsetlayers{background,main}

% define styles for the normal border and the torn border
\tikzset{
  normal border/.style={black!70!gray, decorate, 
     decoration={random steps, segment length=2.5cm, amplitude=.7mm}},
  torn border/.style={black!70!gray, decorate, 
     decoration={random steps, segment length=.5cm, amplitude=1.7mm}}}

% Macro to draw the shape behind the text, when it fits completly in the
% page
\def\parchmentframe#1{
\tikz{
  \node[inner sep=2em] (A) {#1};  % Draw the text of the node
  \begin{pgfonlayer}{background}  % Draw the shape behind
  \fill[normal border] 
        (A.south east) -- (A.south west) -- 
        (A.north west) -- (A.north east) -- cycle;
  \end{pgfonlayer}}}

% Macro to draw the shape, when the text will continue in next page
\def\parchmentframetop#1{
\tikz{
  \node[inner sep=2em] (A) {#1};    % Draw the text of the node
  \begin{pgfonlayer}{background}    
  \fill[normal border]              % Draw the ``complete shape'' behind
        (A.south east) -- (A.south west) -- 
        (A.north west) -- (A.north east) -- cycle;
  \fill[torn border]                % Add the torn lower border
        ($(A.south east)-(0,.2)$) -- ($(A.south west)-(0,.2)$) -- 
        ($(A.south west)+(0,.2)$) -- ($(A.south east)+(0,.2)$) -- cycle;
  \end{pgfonlayer}}}

% Macro to draw the shape, when the text continues from previous page
\def\parchmentframebottom#1{
\tikz{
  \node[inner sep=2em] (A) {#1};   % Draw the text of the node
  \begin{pgfonlayer}{background}   
  \fill[normal border]             % Draw the ``complete shape'' behind
        (A.south east) -- (A.south west) -- 
        (A.north west) -- (A.north east) -- cycle;
  \fill[torn border]               % Add the torn upper border
        ($(A.north east)-(0,.2)$) -- ($(A.north west)-(0,.2)$) -- 
        ($(A.north west)+(0,.2)$) -- ($(A.north east)+(0,.2)$) -- cycle;
  \end{pgfonlayer}}}

% Macro to draw the shape, when both the text continues from previous page
% and it will continue in next page
\def\parchmentframemiddle#1{
\tikz{
  \node[inner sep=2em] (A) {#1};   % Draw the text of the node
  \begin{pgfonlayer}{background}   
  \fill[normal border]             % Draw the ``complete shape'' behind
        (A.south east) -- (A.south west) -- 
        (A.north west) -- (A.north east) -- cycle;
  \fill[torn border]               % Add the torn lower border
        ($(A.south east)-(0,.2)$) -- ($(A.south west)-(0,.2)$) -- 
        ($(A.south west)+(0,.2)$) -- ($(A.south east)+(0,.2)$) -- cycle;
  \fill[torn border]               % Add the torn upper border
        ($(A.north east)-(0,.2)$) -- ($(A.north west)-(0,.2)$) -- 
        ($(A.north west)+(0,.2)$) -- ($(A.north east)+(0,.2)$) -- cycle;
  \end{pgfonlayer}}}

% Define the environment which puts the frame
% In this case, the environment also accepts an argument with an optional
% title (which defaults to ``Example'', which is typeset in a box overlaid
% on the top border
\newenvironment{parchment}[1][Example]{%
  \def\FrameCommand{\parchmentframe}%
  \def\FirstFrameCommand{\parchmentframetop}%
  \def\LastFrameCommand{\parchmentframebottom}%
  \def\MidFrameCommand{\parchmentframemiddle}%
  \vskip\baselineskip
  \MakeFramed {\FrameRestore}
  \noindent\tikz\node[inner sep=1ex, draw=black!20, fill=black!90, 
          anchor=west, overlay] at (0em, 2em) {\sffamily#1};\par}%
{\endMakeFramed}

% ----------------------------------------------

\mode<presentation>{
    \usetheme{Warsaw}
    % Boadilla CambridgeUS
    % default Antibes Berlin Copenhagen
    % Madrid Montpelier Ilmenau Malmoe
    % Berkeley Singapore Warsaw
    \usecolortheme{seagull}
    % beetle, beaver, orchid, whale, dolphin, seagull
    \useoutertheme{infolines}
    % infolines miniframes shadow sidebar smoothbars smoothtree split tree
    \useinnertheme{circles}
    % circles, rectanges, rounded, inmargin
}

% ---------------------------------------------------------------------
% Jet Black Theme
\setbeamercolor{normal text}{fg=white,bg=black!90}
\setbeamercolor{structure}{fg=white}

\setbeamercolor{alerted text}{fg=red!85!black}

\setbeamercolor{item projected}{use=item,fg=black,bg=item.fg!35}

\setbeamercolor*{palette primary}{use=structure,fg=structure.fg}
\setbeamercolor*{palette secondary}{use=structure,fg=structure.fg!95!black}
\setbeamercolor*{palette tertiary}{use=structure,fg=structure.fg!90!black}
\setbeamercolor*{palette quaternary}{use=structure,fg=structure.fg!95!black,bg=black!80}

\setbeamercolor*{framesubtitle}{fg=white}

\setbeamercolor*{block title}{parent=structure,bg=black!70!gray}
\setbeamercolor*{block body}{fg=black,bg=black!10}
\setbeamercolor*{block title alerted}{parent=alerted text,bg=black!15}
\setbeamercolor*{block title example}{parent=example text,bg=black!15}
% ---------------------------------------------------------------------


% ---------------------------------------------------------------------
% flow chart
\tikzset{
    >=stealth',
    punktchain/.style={
        rectangle, 
        rounded corners, 
        % fill=black!10,
        draw=white, very thick,
        text width=6em,
        minimum height=2em, 
        text centered, 
        on chain
    },
    largepunktchain/.style={
        rectangle,
        rounded corners,
        draw=white, very thick,
        text width=10em,
        minimum height=2em,
        on chain
    },
    line/.style={draw, thick, <-},
    element/.style={
        tape,
        top color=white,
        bottom color=blue!50!black!60!,
        minimum width=6em,
        draw=blue!40!black!90, very thick,
        text width=6em, 
        minimum height=2em, 
        text centered, 
        on chain
    },
    every join/.style={->, thick,shorten >=1pt},
    decoration={brace},
    tuborg/.style={decorate},
    tubnode/.style={midway, right=2pt},
    font={\fontsize{10pt}{12}\selectfont},
}
% ---------------------------------------------------------------------

% code setting
\lstset{
    language=C++,
    basicstyle=\ttfamily\footnotesize,
    keywordstyle=\color{red},
    breaklines=true,
    xleftmargin=2em,
    numbers=left,
    numberstyle=\color[RGB]{222,155,81},
    frame=leftline,
    tabsize=4,
    breakatwhitespace=false,
    showspaces=false,               
    showstringspaces=false,
    showtabs=false,
    morekeywords={Str, Num, List},
}

% ---------------------------------------------------------------------

\newcommand{\reditem}[1]{\setbeamercolor{item}{fg=red}\item #1}

% 缩放公式大小
\newcommand*{\Scale}[2][4]{\scalebox{#1}{\ensuremath{#2}}}

% 解决 font warning
\renewcommand\textbullet{\ensuremath{\bullet}}

% -------------------------------------------------------------

%% preamble
\title[Programowanie sieciowe]{Programowanie sieciowe}
% \subtitle{The subtitle}
\author{Adam Djellouli}

% -------------------------------------------------------------

\begin{document}

% title frame
\begin{frame}
	\titlepage
\end{frame}

\begin{frame}
	Protokół TCP
	
	\begin{itemize}
		\item Standard definiujący jak ustanowić i utrzymać połączenie między urządzeniami w sieci.
		\item Działa wspólnie z protokołem IP, który definiuje jak dane w postaci pakietów, przesyłane są między urządzeniami.
		\item Przesyłanie plików i maili przez internet, bez utraty danych możliwe jest dzięki temu protokołowi.
		\item TCP ustanawia połączenie między nadawcą a odbiorcą, zanim dane zostaną przesłane i utrzymuje je w czasie przesyłania danych.
		\end{frame}
		
		\begin{frame}
			Cechy połączenia TCP
			
			\begin{itemize}
				\item Zorientowany na połączenie - nadawca zwraca się z prośbą o uzyskanie zgody na połączenie do odbiorcy.
				\item Protokół typu punkt-punkt - każde połączenie TCP ma dokładnie dwa końce.
				\item Niezawodność - wszystkie dane zostaną przesłane.
				\item Dwukierunkowość.
				\item Strumieniowy interfejs - dane nie muszą być przesłane do odbiorcy w kawałkach o takiej samej wielkości w jakiej zostały wysłane.
				\item Łagodne zakończenie połączenia - pakiety nie zostaną utracone.
			\end{itemize}
		\end{frame}
		
		\begin{frame}
			Model klient-serwer
			\begin{itemize}
				\item Klient wysyła zapytanie o połączenie do serwera (musi znać adres IP oraz port serwera).
				\item Serwer udziela zgodę na połączenie lub odrzuca prośbę.
				\item Klient otrzymuje odpowiedź i wysyła dane do serwera.
				\item Serwer odbiera dane i ewentualnie wysyła odpowiedź.
				\item Połączenie może być zamknięte przez serwer lub klienta.
				\item Jednocześnie z jednym serwerem może być połączonych wielu klientów.
				\item Komunikacja odbywa się poprzez protokoły TCP bądź UDP, zarówno w obrębie LAN jak i WAN.
				\item Klient-serwer to centralna koncepcja w teorii sieci komputerowych.
			\end{itemize}
		\end{frame}
		
		\begin{frame}
			Typy serwerów
			
			Serwery iteracyjne
			\begin{itemize}
				\item W danej chwili może być obsłużony tylko jeden klient.
				\item Program przechodzi przez wszystkich klientów, przetwarzając ich zapytania.
				\item Klienci, którzy nie są w danej chwili obsługiwani, czekają na swoją kolej.
				\item Zaletą tego typu serwerów jest prostota implementacji.
			\end{itemize}
			
			Serwery współbieżne
			\begin{itemize}
				\item W danej chwili może być obsłużonych wielu klientów.
				\item W systemach uniksopodobnych serwer może przy nowym połączeniu użyć funkcji fork, by stworzyć proces pochodny
				      będący kopią oryginalnego serwera.
				\end{frame}
				
				\begin{frame}
					Gniazda
					
					\begin{itemize}
						\item Gniazdo to wirtualny punkt końcowy połączeń w sieci.
						\item Programowanie gniazd polega na ustaleniu reguł komunikacji między dwoma punktami sieci.
						\item Gniazdo może być wykorzystane do nasłuchiwania połączeń na określonym porcie i pod danym adresem IP.
						\item Jeśli inne gniazdo próbuje się połączyć z tym portem i adresem IP, to próba połączenia zostanie obsłużona przez gniazdo nasłuchujące.
						\item Gdy gniazda są ze sobą połączone, to mogą między sobą przesyłać dane.
					\end{itemize}
				\end{frame}
				
				\begin{frame}
					Komunikacja między gniazdami:
					
					\begin{itemize}
						\item Utwórz gniazdo (socket) i przypisz mu adres IP oraz port.
						\item Podejmij próbę połączenia się z innym gniazdem (connect).
						\item Nasłuchuj połączeń (listen).
						\item Zaakceptuj prośbę o połączenie (accept).
						\item Powiąż lokalną nazwę z deskryptorem gniazda (bind).
						\item Wyślij dane do drugiego gniazda (send).
						\item Odbierz dane z drugiego gniazda (recv).
						\item Zakończ połączenie (close).
						      
						\end{frame}
						
						\begin{frame}
							Przegląd bibliotek
							
							<sys/types.h>
							definicje typów używanych w komunikacji między gniazdami
							
							<sys/socket.h>
							definicja struktury reprezentującej gniazdo
							funkcje wykorzystywane do komunikacji między gniazdami (connect, listen, accept, send, recv, close) 
							
							<netinet/in.h>
							stałe oraz struktury reprezentujące adresy IP
							
							<netdb.h>
							definicje operacji sieciowych na bazach danych
							
						\end{frame}
						
						\begin{frame}
							Porty
							
							\begin{itemize}
								\item Na jednym komputerze może jednocześnie być uruchomionych wiele aplikacji komunikujących się z siecią.
								\item Aby uniknąć problemów z konfliktami wprowadzono porty, liczby służące do identyfikacji aplikacji.
								\item Każdy z portów to dwukieronkowa droga, umożliwia zarówno przesyłanie jak i odbieranie informacji.
								\item Porty to liczby całkowite z zakresu 0-65535.
								\item Liczby między 0-1023 są zarezerwowane dla systemów operacyjnych. Telnet używa portu 23, FTP używa portu 21, HTTP używa portu 80 itd.
							\end{itemize}
							
						\end{frame}
						
\end{document}
