%!TeX encoding = UTF-8
%!TeX program = xelatex
\documentclass[notheorems, aspectratio=54]{beamer}
% aspectratio: 1610, 149, 54, 43(default), 32
\usepackage[utf8]{inputenc} % `utf8` option to match Editor encoding
\usepackage[T1]{fontenc}
\usepackage{latexsym}
\usepackage{amsmath,amssymb}
\usepackage{mathtools}
\usepackage{color,xcolor}
\usepackage{graphicx}
\usepackage{algorithm}
\usepackage{amsthm}
\usepackage{lmodern} % 解决 font warning
% \usepackage[UTF8]{ctex}
\usepackage{animate} % insert gif
\usepackage{listings}

\usepackage{karnaugh-map}

\usetikzlibrary{matrix,calc}
\usepackage{lipsum} % To generate test text 
\usepackage{ulem} % 下划线,波浪线

\usepackage{listings} % display code on slides; don't forget [fragile] option after \begin{frame}


% ----------------------------------------------
% tikx
\usepackage{framed}
\usepackage{tikz}
\usepackage{pgf}
\usetikzlibrary{calc,trees,positioning,arrows,chains,shapes.geometric,%
    decorations.pathreplacing,decorations.pathmorphing,shapes,%
    matrix,shapes.symbols}
\pgfmathsetseed{1} % To have predictable results
% Define a background layer, in which the parchment shape is drawn
\pgfdeclarelayer{background}
\pgfsetlayers{background,main}

% define styles for the normal border and the torn border
\tikzset{
  normal border/.style={black!70!gray, decorate, 
     decoration={random steps, segment length=2.5cm, amplitude=.7mm}},
  torn border/.style={black!70!gray, decorate, 
     decoration={random steps, segment length=.5cm, amplitude=1.7mm}}}

% Macro to draw the shape behind the text, when it fits completly in the
% page
\def\parchmentframe#1{
\tikz{
  \node[inner sep=2em] (A) {#1};  % Draw the text of the node
  \begin{pgfonlayer}{background}  % Draw the shape behind
  \fill[normal border] 
        (A.south east) -- (A.south west) -- 
        (A.north west) -- (A.north east) -- cycle;
  \end{pgfonlayer}}}

% Macro to draw the shape, when the text will continue in next page
\def\parchmentframetop#1{
\tikz{
  \node[inner sep=2em] (A) {#1};    % Draw the text of the node
  \begin{pgfonlayer}{background}    
  \fill[normal border]              % Draw the ``complete shape'' behind
        (A.south east) -- (A.south west) -- 
        (A.north west) -- (A.north east) -- cycle;
  \fill[torn border]                % Add the torn lower border
        ($(A.south east)-(0,.2)$) -- ($(A.south west)-(0,.2)$) -- 
        ($(A.south west)+(0,.2)$) -- ($(A.south east)+(0,.2)$) -- cycle;
  \end{pgfonlayer}}}

% Macro to draw the shape, when the text continues from previous page
\def\parchmentframebottom#1{
\tikz{
  \node[inner sep=2em] (A) {#1};   % Draw the text of the node
  \begin{pgfonlayer}{background}   
  \fill[normal border]             % Draw the ``complete shape'' behind
        (A.south east) -- (A.south west) -- 
        (A.north west) -- (A.north east) -- cycle;
  \fill[torn border]               % Add the torn upper border
        ($(A.north east)-(0,.2)$) -- ($(A.north west)-(0,.2)$) -- 
        ($(A.north west)+(0,.2)$) -- ($(A.north east)+(0,.2)$) -- cycle;
  \end{pgfonlayer}}}

% Macro to draw the shape, when both the text continues from previous page
% and it will continue in next page
\def\parchmentframemiddle#1{
\tikz{
  \node[inner sep=2em] (A) {#1};   % Draw the text of the node
  \begin{pgfonlayer}{background}   
  \fill[normal border]             % Draw the ``complete shape'' behind
        (A.south east) -- (A.south west) -- 
        (A.north west) -- (A.north east) -- cycle;
  \fill[torn border]               % Add the torn lower border
        ($(A.south east)-(0,.2)$) -- ($(A.south west)-(0,.2)$) -- 
        ($(A.south west)+(0,.2)$) -- ($(A.south east)+(0,.2)$) -- cycle;
  \fill[torn border]               % Add the torn upper border
        ($(A.north east)-(0,.2)$) -- ($(A.north west)-(0,.2)$) -- 
        ($(A.north west)+(0,.2)$) -- ($(A.north east)+(0,.2)$) -- cycle;
  \end{pgfonlayer}}}

% Define the environment which puts the frame
% In this case, the environment also accepts an argument with an optional
% title (which defaults to ``Example'', which is typeset in a box overlaid
% on the top border
\newenvironment{parchment}[1][Example]{%
  \def\FrameCommand{\parchmentframe}%
  \def\FirstFrameCommand{\parchmentframetop}%
  \def\LastFrameCommand{\parchmentframebottom}%
  \def\MidFrameCommand{\parchmentframemiddle}%
  \vskip\baselineskip
  \MakeFramed {\FrameRestore}
  \noindent\tikz\node[inner sep=1ex, draw=black!20, fill=black!90, 
          anchor=west, overlay] at (0em, 2em) {\sffamily#1};\par}%
{\endMakeFramed}

% ----------------------------------------------

\mode<presentation>{
    \usetheme{Warsaw}
    % Boadilla CambridgeUS
    % default Antibes Berlin Copenhagen
    % Madrid Montpelier Ilmenau Malmoe
    % Berkeley Singapore Warsaw
    \usecolortheme{seagull}
    % beetle, beaver, orchid, whale, dolphin, seagull
    \useoutertheme{infolines}
    % infolines miniframes shadow sidebar smoothbars smoothtree split tree
    \useinnertheme{circles}
    % circles, rectanges, rounded, inmargin
}

% ---------------------------------------------------------------------
% Jet Black Theme
\setbeamercolor{normal text}{fg=white,bg=black!90}
\setbeamercolor{structure}{fg=white}

\setbeamercolor{alerted text}{fg=red!85!black}

\setbeamercolor{item projected}{use=item,fg=black,bg=item.fg!35}

\setbeamercolor*{palette primary}{use=structure,fg=structure.fg}
\setbeamercolor*{palette secondary}{use=structure,fg=structure.fg!95!black}
\setbeamercolor*{palette tertiary}{use=structure,fg=structure.fg!90!black}
\setbeamercolor*{palette quaternary}{use=structure,fg=structure.fg!95!black,bg=black!80}

\setbeamercolor*{framesubtitle}{fg=white}

\setbeamercolor*{block title}{parent=structure,bg=black!70!gray}
\setbeamercolor*{block body}{fg=black,bg=black!10}
\setbeamercolor*{block title alerted}{parent=alerted text,bg=black!15}
\setbeamercolor*{block title example}{parent=example text,bg=black!15}
% ---------------------------------------------------------------------


% ---------------------------------------------------------------------
% flow chart
\tikzset{
    >=stealth',
    punktchain/.style={
        rectangle, 
        rounded corners, 
        % fill=black!10,
        draw=white, very thick,
        text width=6em,
        minimum height=2em, 
        text centered, 
        on chain
    },
    largepunktchain/.style={
        rectangle,
        rounded corners,
        draw=white, very thick,
        text width=10em,
        minimum height=2em,
        on chain
    },
    line/.style={draw, thick, <-},
    element/.style={
        tape,
        top color=white,
        bottom color=blue!50!black!60!,
        minimum width=6em,
        draw=blue!40!black!90, very thick,
        text width=6em, 
        minimum height=2em, 
        text centered, 
        on chain
    },
    every join/.style={->, thick,shorten >=1pt},
    decoration={brace},
    tuborg/.style={decorate},
    tubnode/.style={midway, right=2pt},
    font={\fontsize{10pt}{12}\selectfont},
}
% ---------------------------------------------------------------------

% code setting
\lstset{
    language=C++,
    basicstyle=\ttfamily\footnotesize,
    keywordstyle=\color{red},
    breaklines=true,
    xleftmargin=2em,
    numbers=left,
    numberstyle=\color[RGB]{222,155,81},
    frame=leftline,
    tabsize=4,
    breakatwhitespace=false,
    showspaces=false,               
    showstringspaces=false,
    showtabs=false,
    morekeywords={Str, Num, List},
}

% ---------------------------------------------------------------------

\newcommand{\reditem}[1]{\setbeamercolor{item}{fg=red}\item #1}

% 缩放公式大小
\newcommand*{\Scale}[2][4]{\scalebox{#1}{\ensuremath{#2}}}

% 解决 font warning
\renewcommand\textbullet{\ensuremath{\bullet}}

% -------------------------------------------------------------

%% preamble
\title[Wskaźniki]{Wskaźniki}
% \subtitle{The subtitle}
\author{Adam Djellouli}

% -------------------------------------------------------------

\begin{document}

% title frame
\begin{frame}
    \titlepage
\end{frame}

\begin{frame}

\begin{itemize}
  \item Sygnał to krótka wiadomość, która zawiera jedynie jedną liczbę całkowitą.
  \item Sygnały są jednym z rodzajów komunikacji między procesami (IPC) w systemach uniksopodobnych.
  \item Sygnały są generowane przez system, ale użytkownik może również generować sygnały programistycznie.
  \item Procesy mogą otrzymywać sygnały asynchronicznie (w dowolnym momencie wykonywania).
  \item Po otrzymaniu sygnału, proces musi zatrzymać swoją obecną pracę i przejść do obsługi sygnału (więcej na ten temat w sekcji domyślne działania).
  \item Dostępne sygnały są definiowane przez system operacyjny.
\end{frame}

\begin{frame}

Przykłady:

\begin{itemize}
  \item Jeśli chcemy zakończyć wykonywanie wszystkich instancji programu firefox, możemy użyć komendy: killall firefox.
  \item System operacyjny znajdzie wszystkie instancje programu firefox i wyśle do nich sygnał SIGTERM.
  \item Gdy w terminalu użyjesz komendy Ctrl-c, system operacyjny wyśle sygnał SIGINT do aktualnie wykonywanego programu.
  \item Natomiast gdy użyjesz komendy Ctrl-z, system operacyjny wyśle sygnał SIGTSTP do aktualnie wykonywanego programu. 
  \item Po wywołaniu funkcji fgets() przez program, otrzymuje on sygnał SIGIO oznaczający, że dane są gotowe do odczytu.
\end{itemize}
\end{frame}

\begin{frame}
  Po otrzymaniu sygnału proces ma trzy możliwości:

  \begin{enumerate}
    \item Wykonać domyślne działanie.
    \item Obsłużyć sygnał.
    \item Zignorować sygnał (nie wszystkie sygnały są ignorowalne).
  \end{enumerate}

\end{frame}

\begin{frame}

Komunikacja między procesami
\begin{itemize}
  \item Każdy proces ma własną przestrzeń adresów.
  \item Jeśli jeden proces chce przekazać jakąś informację do innego procesu, to odbywa się to poprzez jeden z mechnizmów IPC.
  \item Komunikacja między procesami może być widoczna jako metoda współpracy pomiędzy procesami.
  \item Komunikacja może odbywać się zarówno między spokrewnionymi procesami, jak i między procesami, które nie są spokrewnione.
\end{itemize}

\end{frame}

\begin{frame}

Generowanie sygnałów

\begin{itemize}
  \item Biblioteka <signal.h> zawiera funkcje, które pozwalają na wywołanie sygnałów.
  \item Sygnały są generowane przez wywołanie funkcji raise() lub kill().
  \item Funkcja raise(int numer) wysyła sygnał o danym numerze do aktualnie wykonywanego programu.
  \item Funkcja kill() wysyła sygnał do procesu o podanym identyfikatorze.
  \item Jeśli funkcja raise() zakończy się powodzeniem, to zwrócone zostanie 0.
  \item Po wysłaniu sygnału, wykonywanie aktualnego procesu zostaje zatrzymane.
\end{itemize}  

\begin{lstlisting}[language=C++]
#include <signal.h>
#include <stdio.h>

  int main(int argc, char *argv[])
  {
    // wysłanie sygnału SIGINT do aktualnie wykonywanego programu
    raise(SIGINT);
   
    // ta linijka nie zostanie wykonana
    printf("Wysłano sygnał SIGINT \n");

    return 0;
  }
\end{lstlisting}

\end{frame}

\begin{frame}
Obsługa sygnałów

\begin{itemize}
  \item Aby obsłużyć sygnał, należy zaimplementować funkcję obsługi sygnału.
  \item Funkcję obsługi sygnału należy przekazać jako drugi parametr do funkcji signal(numer_sygnału, funkcja_obsługi).
  \item Funkcja signal() zwraca wskaźnik do funkcji obsługi sygnału.
  \item W funkcji obsługi sygnału można wykonać dowolne działanie.
  \item W funkcji obsługi sygnału nie można wywołać funkcji raise() lub kill().
\end{itemize}

\begin{lstlisting}[language=C++]
#include <signal.h>
#include <stdio.h>

  void handler(int signum)
  {
    printf("Wywołano obsługę sygnału %d \n", signum);
  }

  int main(int argc, char *argv[])
  {
    // obsługa sygnału SIGINT
    signal(SIGINT, handler);

    // ta linijka nie zostanie wykonana
    printf("Wysłano sygnał SIGINT \n");

    return 0;
  }
\end{lstlisting}
\end{frame}  

\end{document}