%!TeX encoding = UTF-8
%!TeX program = xelatex
\documentclass[notheorems, aspectratio=54]{beamer}
% aspectratio: 1610, 149, 54, 43(default), 32
\usepackage[utf8]{inputenc} % `utf8` option to match Editor encoding
\usepackage[T1]{fontenc}
\usepackage{latexsym}
\usepackage{amsmath,amssymb}
\usepackage{mathtools}
\usepackage{color,xcolor}
\usepackage{graphicx}
\usepackage{algorithm}
\usepackage{amsthm}
\usepackage{lmodern} % 解决 font warning
% \usepackage[UTF8]{ctex}
\usepackage{animate} % insert gif
\usepackage{listings}

\usepackage{karnaugh-map}

\usetikzlibrary{matrix,calc}
\usepackage{lipsum} % To generate test text 
\usepackage{ulem} % 下划线,波浪线

\usepackage{listings} % display code on slides; don't forget [fragile] option after \begin{frame}


% ----------------------------------------------
% tikx
\usepackage{framed}
\usepackage{tikz}
\usepackage{pgf}
\usetikzlibrary{calc,trees,positioning,arrows,chains,shapes.geometric,%
decorations.pathreplacing,decorations.pathmorphing,shapes,%
matrix,shapes.symbols}
\pgfmathsetseed{1} % To have predictable results
% Define a background layer, in which the parchment shape is drawn
\pgfdeclarelayer{background}
\pgfsetlayers{background,main}

% define styles for the normal border and the torn border
\tikzset{
normal border/.style={black!70!gray, decorate, 
 decoration={random steps, segment length=2.5cm, amplitude=.7mm}},
torn border/.style={black!70!gray, decorate, 
 decoration={random steps, segment length=.5cm, amplitude=1.7mm}}}

% Macro to draw the shape behind the text, when it fits completly in the
% page
\def\parchmentframe#1{
\tikz{
\node[inner sep=2em] (A) {#1};  % Draw the text of the node
\begin{pgfonlayer}{background}  % Draw the shape behind
\fill[normal border] 
    (A.south east) -- (A.south west) -- 
    (A.north west) -- (A.north east) -- cycle;
\end{pgfonlayer}}}

% Macro to draw the shape, when the text will continue in next page
\def\parchmentframetop#1{
\tikz{
\node[inner sep=2em] (A) {#1};    % Draw the text of the node
\begin{pgfonlayer}{background}    
\fill[normal border]              % Draw the ``complete shape'' behind
    (A.south east) -- (A.south west) -- 
    (A.north west) -- (A.north east) -- cycle;
\fill[torn border]                % Add the torn lower border
    ($(A.south east)-(0,.2)$) -- ($(A.south west)-(0,.2)$) -- 
    ($(A.south west)+(0,.2)$) -- ($(A.south east)+(0,.2)$) -- cycle;
\end{pgfonlayer}}}

% Macro to draw the shape, when the text continues from previous page
\def\parchmentframebottom#1{
\tikz{
\node[inner sep=2em] (A) {#1};   % Draw the text of the node
\begin{pgfonlayer}{background}   
\fill[normal border]             % Draw the ``complete shape'' behind
    (A.south east) -- (A.south west) -- 
    (A.north west) -- (A.north east) -- cycle;
\fill[torn border]               % Add the torn upper border
    ($(A.north east)-(0,.2)$) -- ($(A.north west)-(0,.2)$) -- 
    ($(A.north west)+(0,.2)$) -- ($(A.north east)+(0,.2)$) -- cycle;
\end{pgfonlayer}}}

% Macro to draw the shape, when both the text continues from previous page
% and it will continue in next page
\def\parchmentframemiddle#1{
\tikz{
\node[inner sep=2em] (A) {#1};   % Draw the text of the node
\begin{pgfonlayer}{background}   
\fill[normal border]             % Draw the ``complete shape'' behind
    (A.south east) -- (A.south west) -- 
    (A.north west) -- (A.north east) -- cycle;
\fill[torn border]               % Add the torn lower border
    ($(A.south east)-(0,.2)$) -- ($(A.south west)-(0,.2)$) -- 
    ($(A.south west)+(0,.2)$) -- ($(A.south east)+(0,.2)$) -- cycle;
\fill[torn border]               % Add the torn upper border
    ($(A.north east)-(0,.2)$) -- ($(A.north west)-(0,.2)$) -- 
    ($(A.north west)+(0,.2)$) -- ($(A.north east)+(0,.2)$) -- cycle;
\end{pgfonlayer}}}

% Define the environment which puts the frame
% In this case, the environment also accepts an argument with an optional
% title (which defaults to ``Example'', which is typeset in a box overlaid
% on the top border
\newenvironment{parchment}[1][Example]{%
\def\FrameCommand{\parchmentframe}%
\def\FirstFrameCommand{\parchmentframetop}%
\def\LastFrameCommand{\parchmentframebottom}%
\def\MidFrameCommand{\parchmentframemiddle}%
\vskip\baselineskip
\MakeFramed {\FrameRestore}
\noindent\tikz\node[inner sep=1ex, draw=black!20, fill=black!90, 
      anchor=west, overlay] at (0em, 2em) {\sffamily#1};\par}%
{\endMakeFramed}

% ----------------------------------------------

\mode<presentation>{
\usetheme{Warsaw}
% Boadilla CambridgeUS
% default Antibes Berlin Copenhagen
% Madrid Montpelier Ilmenau Malmoe
% Berkeley Singapore Warsaw
\usecolortheme{seagull}
% beetle, beaver, orchid, whale, dolphin, seagull
\useoutertheme{infolines}
% infolines miniframes shadow sidebar smoothbars smoothtree split tree
\useinnertheme{circles}
% circles, rectanges, rounded, inmargin
}

% ---------------------------------------------------------------------
% Jet Black Theme
\setbeamercolor{normal text}{fg=white,bg=black!90}
\setbeamercolor{structure}{fg=white}

\setbeamercolor{alerted text}{fg=red!85!black}

\setbeamercolor{item projected}{use=item,fg=black,bg=item.fg!35}

\setbeamercolor*{palette primary}{use=structure,fg=structure.fg}
\setbeamercolor*{palette secondary}{use=structure,fg=structure.fg!95!black}
\setbeamercolor*{palette tertiary}{use=structure,fg=structure.fg!90!black}
\setbeamercolor*{palette quaternary}{use=structure,fg=structure.fg!95!black,bg=black!80}

\setbeamercolor*{framesubtitle}{fg=white}

\setbeamercolor*{block title}{parent=structure,bg=black!70!gray}
\setbeamercolor*{block body}{fg=black,bg=black!10}
\setbeamercolor*{block title alerted}{parent=alerted text,bg=black!15}
\setbeamercolor*{block title example}{parent=example text,bg=black!15}
% ---------------------------------------------------------------------


% ---------------------------------------------------------------------
% flow chart
\tikzset{
>=stealth',
punktchain/.style={
    rectangle, 
    rounded corners, 
    % fill=black!10,
    draw=white, very thick,
    text width=6em,
    minimum height=2em, 
    text centered, 
    on chain
},
largepunktchain/.style={
    rectangle,
    rounded corners,
    draw=white, very thick,
    text width=10em,
    minimum height=2em,
    on chain
},
line/.style={draw, thick, <-},
element/.style={
    tape,
    top color=white,
    bottom color=blue!50!black!60!,
    minimum width=6em,
    draw=blue!40!black!90, very thick,
    text width=6em, 
    minimum height=2em, 
    text centered, 
    on chain
},
every join/.style={->, thick,shorten >=1pt},
decoration={brace},
tuborg/.style={decorate},
tubnode/.style={midway, right=2pt},
font={\fontsize{10pt}{12}\selectfont},
}
% ---------------------------------------------------------------------

% code setting
\lstset{
language=C++,
basicstyle=\ttfamily\footnotesize,
keywordstyle=\color{red},
breaklines=true,
xleftmargin=2em,
numbers=left,
numberstyle=\color[RGB]{222,155,81},
frame=leftline,
tabsize=4,
breakatwhitespace=false,
showspaces=false,               
showstringspaces=false,
showtabs=false,
morekeywords={Str, Num, List},
}

% ---------------------------------------------------------------------

\newcommand{\reditem}[1]{\setbeamercolor{item}{fg=red}\item #1}

% 缩放公式大小
\newcommand*{\Scale}[2][4]{\scalebox{#1}{\ensuremath{#2}}}

% 解决 font warning
\renewcommand\textbullet{\ensuremath{\bullet}}

% -------------------------------------------------------------

%% preamble
\title[Zjawisko wyscigu]{Zjawisko wyscigu}
% \subtitle{The subtitle}
\author{Adam Djellouli}

% -------------------------------------------------------------

\begin{document}

% title frame
\begin{frame}
\titlepage
\end{frame}

\begin{frame}

Zjawisko wyścigu

\begin{itemize}
\item Podczas pracy z wątkami musimy być szczególnie ostrożni, gdyż kolejność w jakiej ułożyliśmy instrukcje w kodzie może nie pokrywać się z kolejnością w jakiej instrukcje zostaną wykonane. 
\item Wątki mogą być wykonywane w dowolnej kolejności.
\item Prędkość wykonywania wątków może się różnić.
\item Kilka wątków może wykonywać operacje na tym samym obszarze pamięci.
\item Jeśli te operacje są zależne od kolejności ich wykonania, to taką sytuację zwiemy zjawiskiem wyścigu.
\item Przykładowo jeśli jedną zmienną próbujemy odczytać i nadpisać z różnych wątków, to próba nadpisania może pokryć się z oczekiwaniami na odczytanie.
\end{itemize}

Zjawisko wyścigu (race condition) występuje gdy wątki współbieżnie korzystają z zasobu dzielonego (np. zmiennej), przy czym przynajmniej jeden z nich zmienia stan tego zasobu.
Wyścig często jest przyczyną niedeterministycznego zachowania się programu i może prowadzić do trudnych do wykrycia błędów.
Pojęcie thread-safety (bezpieczeństwo dla wątków, wielobieżność): kod jest bezpieczny dla wątków, gdy jego współbieżne wykonanie nie powoduje sytuacji wyścigu. 

\end{frame}

\begin{frame}
Cztery typy programowania wielowątkowego:

\begin{itemize}
\item Niezależne wątki. Najprostszy model: wątki niepowiązane ze sobą. Mechanizmy programistyczne: podstawowe abstrakcje do tworzenia wątków.
\item Powiązane, niesynchronizowane wątki. Wątki powiązane ze sobą, jednak nie oddziałują na siebie i nie posiadają danych współdzielonych. Np. problem podzielony na podproblemy, z których każdy przetwarzany jest przez oddzielny wątek. Mechanizmy programistyczne: pule wątków.
\item Wątki wzajemnie wykluczające się. Wątki operują na współdzielonych danych i muszą być synchronizowane. Np. wątki czytające i modyfikujące stan zmiennej. Mechanizmy programistyczne: synchronizacja wątków (semafory, blokady etc.)
\item Komunikujące się, wzajemnie wykluczające wątki. Wątki operują na współdzielonych danych, ale również komunikują się / koordynują działania. Np. zależność w stylu producent-konsument. Mechanizmy programistyczne: koordynacja typu wait/notify, zmienne warunkowe.
\end{itemize}

\end{frame}

\begin{frame}
Zakleszczenie
\begin{itemize}
\item Zakleszczenie jest sytuację w której wątek wykonujący operację zostaje zablokowany przez inny wątek.
\item Problem pięciu filozofów jest jednym z klasycznych problemów teorii współbieżności.
\item N filozofów zasiada przy okrągłym stole
\item Pomiędzy sąsiednimi filozofami leży widelec (łącznie jest N widelców)
\item Każdy filozof działa ciągle według schematu „myślenie - jedzenie - myślenie - jedzenie - ... ”.
\item Aby zjeść, filozof musi podnieść oba sąsiadujące widelce
\end{itemize}

\begin{enumerate}
\item Naiwne rozwiązanie. Każdy filozof czeka najpierw na widelec z lewej strony, a potem na widelec 
      z prawej strony. Wadą tego rozwiązania jest możliwość pojawienia się blokady.
\item Część filozofów umiera z głodu, ale nie ma blokady. Każdy filozof sprawdza, czy 
      wolne są oba widelce i je tylko, gdy ma to miejsce. Jeśli, któryś widelec jest stale 
      zajęty filozof umiera z głodu.
\item Rozwiązanie asymetryczne. Filozofowie są ponumerowani. Filozof z parzystym numerem
      najpierw podnosi prawy widelec, filozof z nieparzystym numerem najpierw podnosi lewy
      widelec.
\item Rozwiązanie z arbitrem. Zewnętrzny arbiter pilnuje, aby jednocześnie co
      najwyżej czterech filozofów konkurowało o widelce. Jeśli naraz
      wszyscy filozofowie będą chcieli jeść, arbiter powstrzymuje jednego z nich aż do czasu, gdy
      któryś z filozofów skończy jeść.
\end{enumerate}

\end{frame}

\begin{frame}
Kod wątkowo bezpieczny.
\begin{itemize}
\item Ze względu na problemy ze zjawiskiem wyścigu oraz zakleszczenie funkcja wywoływana w wątku musi być wątkowo bezpieczna.
\item Kod tylko manipuluje współdzielonymi strukturami danych w taki sposób, aby wszystkie wątki działały prawidłowo bez niepowodzenia.
\item Wątkowo bezpieczny oznacza, że nie ma statycznych lub globalnych zmiennych (współdzielonych zasobów) które inne wątki mogą uszkodzić lub zniszczyć.
\item Wątkowo bezpieczny oznacza, że program chroni współdzielone dane.
\item Prawdopodobnie poprzez użycie współdzielonych zasobów, wątkowo bezpieczne funkcje mogą być używane tylko przez jeden wątek w programie i unikalność wątku musi zostać zapewniona.
\end{itemize}
\end{frame}

\begin{frame}
Wielowątkowość w C
\begin{itemize}
\item Domyślnie w C mamy jeden wątek, który jest głównym wątkiem (main).
\item C nie ma wbudowanej obsługi wielowątkowości.
\item Dopiero w standardzie C11 mamy wsparcie dla wielowątkowości.
\item Wielowątkowość tradycyjnie dostępna była dzięki sprzętowi i systemowi operacyjnemu.
\item Implementacje różniły się między sobą, dlatego pisanie wielowątkowych aplikacji przenośnych między różnymi platformami.
\item W celu uzyskania pełnego wykorzystania możliwości wielowątkowości, wymagano standardowego interfejsu programistycznego. 
\item W 1995, POSIX stał się standardem dla wielu systemów operacyjnych, włączając w to obsługę wielowątkowości.
\item Obecnie piszemy wielowątkowe programy wykorzystując interfejs POSIX.
\item Także nazywamy wielowątkowość pthreads, implementacja jest dostępna za pomocą kompilatora gcc. 
\end{itemize}

\end{frame}

\begin{frame}
Muteksy
\begin{itemize}
\item Muteksy są sposobem synchronizacji dostępu do zasobów współdzielonych.
\item Gdy chronimy dane współdzielone, to jest ona odpowiedzialnością programisty, aby upewnić się, że każdy wątek, który potrzebuje użycia muteksu, to zrobi to.
\item Zbyt często wykonywane jest przez wątek właściciela muteksu aktualizacja globalnych zmiennych.
\item Niektóre zastosowania muteksów są typowymi sekwencjami wykonywanymi przez wątki.
\item W 1995 roku POSIX stał się standardem dla wielu systemów operacyjnych w UNIX, w tym środowisku wątków.
\item Jesteśmy przygotowani na pisania wielowątkowych programów C za pomocą standardu POSIX wątków.
\item Także znany jest jako pthreads, implementacja jest dostępna za pomocą kompilatora gcc.
\end{itemize}

\end{frame}

\end{document}
